\documentclass{article}
\usepackage{graphicx}
\usepackage{multirow}
\usepackage{caption}
\usepackage[utf8]{inputenc}
  \usepackage[
    backend=biber,
    style=ieee,
  ]{biblatex}
 


\title{CS6000 - Ninth Assignment}
\author{Marc Moreno Lopez}
\date{November 12th 2018}

\begin{document}

\maketitle

\section{Report}

%DThis journal should discuss your learning in using AWS plus addressing your final paper based on reviews. 

The AWS assignment was really helpful for me. Before this assignment, I had never used any cloud computing tool like Google cloud or AWS. After completing this assignment I've seen that it is a really easy and convenient tool to use and it seem reliable. As for the resources that I used, besides everything Dr. Boult explained in class, which I had to rewatch before doing the assignment, I also followed a YouTube video to setup and log in the remote machine and to explore the available option when working with AWS. Once I had set up and logged in the remote machine, I had some issues since pip for python wasn't setup. After I set it up, it was pretty easy since I just had to install the libraries that I needed. Once I had everything setup, I tired different tools to transfer files. I ended up using Cyberduck, which is a tool that I use daily, but I also experimented with GitHub just to see if there was any difference on how it usually works. 

As for the final version of the paper, Bill and I had a meeting and discussed all the reviews. We agreed that we had to make a few minor changes concerning acronyms and some of the figure captions. The biggest thing we had to work on was on the story. We have tried to polish it and make it smoother. We made several changes to the introduction and the conclusions. We have tried to address as many comments as we could, without changing the essence of the paper. We highly appreciate all the comments in the reviews because we have seen some issues that we hadn't seen. 

Inside my repo I have cloned the overleaf project. In this folder you can find all the files for the tex file.


\end{document}
